\section*{Lecture 9 (02.05.2016}
\subsection*{Statistische Lerntheorie für binäre Klassifikation}

\begin{enumerate}[1.]

\item \textbf{Binäres Klassifikationsproblem}:

\begin{itemize}
\item $X$ Menge der Covariates (Bislang/typisch $\mathbb{R}^n$)
\item $\{-1,1\}$ Variates
\end{itemize}

\textbf{Generatives Modell}: Wahrscheinlichkeitsdichte $\varphi$ auf $X \times X$\\
$\varphi$ ist Orakel von dem wir $x, y$-Paare erfragen können.

\item \textbf{Daten/Sample}: $(x^{(1)},y^{(1)}) , \dots ,(x^{(m)}, y^{(m)})$\\
$m$-Datenpunkte, die unabhängig gemäß $\varphi$ gezogen wurden.

\item \textbf{Klassifikatoren}: Klassifikator $h: X \rightarrow Y, \quad h\in F \subseteq \{f: X \rightarrow Y\}$\\
\textbf{Ziel}: Gegeben die Datenpunkte, wähle guten Klassifikator aus $F$ aus. Bsp.: Logistische \textbf{Regression}:
\[F = \{h_\Theta: \Theta \in \mathbb{R}^{n+1}\}\]
Aus den Daten haben wir dann das beste $\Theta$ bzw. den besten Klassifikator bestimmt.
\item \textbf{Loss-Funktion}: 
\[L = F \times X \times Y \rightarrow [0,1]\]
\[(h,x,y) \mapsto \frac{1}{2} |h(x)-y| = \begin{cases}1,& h(x) \neq y \\2,& h(x) = y\end{cases}\]
\textbf{Empirischer Loss}:
\[L_s(h) = \Sigma_{i=1}^m L(h,x^{(i)},y^{(i)})\]
\[s = \{(x^{(1)}, y^{(1)}) , \dots ,(x^{(m)}, y^{(m)})\} \text{Sample}\]
\textbf{Erwarteter Loss}:
\[\hat{L}(h) = \Sigma_{y \in Y} \int_X L(h,x,y) \varphi(x,y) dx\]
\textbf{Ziel}: Wähle Klassifikator mit möglichst kleinen erwarteten loss.\\
\textbf{Problem}: Wir können den erwarteten Loss nicht ausrechnen da wir $\varphi$ nicht kennen. Den empirischen Loss können wir ausrechnen.

%bild

Klassifikator mit empirischen Loss $ = 0$, aber kein gute Prediktor.\\
\textbf{Idee}: Schränke die Klasse $F$ der potentiellen Klassifikatoren systematisch ein.

\end{enumerate}

\subsection*{Rademacher-Komplexität von Funktionenklassen}
\begin{enumerate}[1.]
\item $Z$ ist Menge; $F \subseteq \{f: Z \rightarrow \mathbb{R}\}$
\item $\varphi$ Wahrscheinlichkeitsdichte auf $Z$. Wir können $\varphi$ anfragen, um Datenpunkte aus $Z$ zu bekommen.
\item \textbf{Sample}: $z^{(1)}, \dots , z^{(m)}$ unabhängig von einander gemäß $\varphi$ aus $Z$ gezogen.
\item Zufällige Label $\sigma^{(1)}, \dots , \sigma^{(m)} \in \{-1,1\}$ zufällig gleichverteilt aus $\{-1,1\}$ gezogen\\
Empirische Rademacher-Komplexität:
\[R_s(F) = E_\sigma \left[ \underset{t \in F}{\sup}  | \frac{2}{m} \Sigma^m_{i=1} \sigma^{(i)} f(z^{(i)}) \right] \]
\[s= \{z^{(1)}, \dots , z^{(m)}\}\]
Erwartete Rademacher-Komplexität:
\[\hat{R}(F) = E_s [R_s(F)]\]
$s$ hängt linear von $\varphi$ ab. D.h. kann nicht berechnet werden, da wir $\varphi$ nicht kennen.
\end{enumerate}

Rademacher Komplexität misst die \textbf{Kapazität} einer Funktionsklasse $F$ positiv mit zufälligen Labeln $\sigma$ zu korrelieren. Wichtiges Theorem zum arbeiten mit Rademann-Komplexität:

\subsection*{Mc Diarmid's Theorem}
Seien $x_1, \dots , x_n$ Zufallsvariablen, die Werte in einer Menge $A$ annehmen. Sei $f: A^m  \rightarrow \mathbb{R}$ eine Funktion, sodass \[ \left( \underset{x_1, \dots , x_n , x_i}{\sup} | f(x_1, \dots , x_n) - f(x_1, \dots , x_{i-1}, x_i, x_{i+1}, \dots , x_n) \right) \subseteq c_i \]
\begin{addmargin}[2 cm]{0 cm}
d.h. $f$ ist von beschränkter Variation in jeder Koordinate $x_i$.
\end{addmargin}
dann gilt für alle $\epsilon > 0$
\begin{enumerate}[(1)]
\item $P[f(x_1, \dots , x_n) - E[f(x_1, \dots, x_m)] \geq \epsilon] \leq \exp \left( - \frac{2 \epsilon^2}{\Sigma^m_{i=1} c_i^2}\right)$
\item $P[E[f(x_1, \dots, x_m)] -f(x_1, \dots, x_m) \geq ] \leq  \exp \left( - \frac{2 \epsilon^2}{\Sigma^m_{i=1} c_i^2}\right)$
\end{enumerate}

\subsection*{Hauptsatz der Rademacher Theorie}
Satz: Seien $\delta \in (0,1), \quad F \subseteq \{f: Z \rightarrow [0,1]\}, \varphi$ eine Wahrscheinlichkeitsdichte auf $Z$ und $s= \{z^{(1)}, \dots , z^{(m)}\}$ ein Sample das gemäß $\varphi$ unabhängig aus $Z$ gezogen wurde. Dann gilt mit Wahrscheinlichkeit $1-\delta$ für jeden $f \in F$, dass
\[\underbrace{E[f(z)]}_{\substack{z \text{zuf. gemäß} \varphi}} = \underbrace{\frac{1}{m} \Sigma^m_{i=1} f(z^{(i)})}_{\substack{\text{können wir ausrechnen}}} + \underbrace{R_s(F)}_{\substack{\text{können wir ausrechnen}}} + \underbrace{3 \sqrt{\frac{m(\frac{2}{\delta})}{2n}}}_{\substack{\text{ wird kleiner, falls }  m \text{ groß wird, oder falls } 1-\epsilon\text{ klein ist }}}\]
Anwendung: Sei $F \subseteq \{h: X \rightarrow Y\}$ mögliche Klassifikation und $F' = \{F \circ (h, id): X \times Y \rightarrow [0,1] | h \in F\}$ dann gilt für alle $h \in f$
\begin{framed}
\[ \underbrace{E[L(h)]}_{\substack{\hat{L}(h)}} \leq \underbrace{L_s(h)}_{\substack{(1)}} + \underbrace{R_s(F')}_{\substack{(2)}} +3\sqrt{\frac{\ln (\frac{2}{\delta})}{2m}}\]
\end{framed}
Mit Wahrscheinlichkeit mindestens $1-\delta$ für Sample\\
$s = \{(x^{(1)},y^{(1)})\}$ das unabhängig gemäß $\phi$ gezogen wurde\\
Falls $(1)$ klein wird, dann ist $(2)$ groß und umgekehrt (Trade-Off-Problem)\\\\
\subsubsection*{Beweis}
\[E [f(z)] \leq \frac{1}{m} \Sigma^m_{i=1} f(z^{(i)}) + \underbrace{\underset{g \in F}{\sup} (E[g(z)] - \frac{1}{m} \Sigma^m_{i=1} g(z^{(i)}))}_{\substack{(*)}}\]
Mc Diarmids Theorem gilt mit Wahrscheinlichkeit mindestens $1-\delta$
\[(*) \quad \underset{g \in F}{\sup}(E[g(z)] - \frac{1}{m} \Sigma^m_{i=1} g(z^{(i)})) \leq\]
\[\underbrace{E_s[\underset{g\in F}{\sup}(E[g(z)] - \frac{1}{m} \Sigma^m_{i=1}g(z^{(i)}))]}_{\substack{(**)}} + \sqrt{\frac{\ln(\frac{2}{\delta})}{2m}}\]
Hier haben wir definiert:
\[\exp(- \frac{2\epsilon^2}{\Sigma^m_{i=1} c_i^2}) = \exp(- \frac{2 \epsilon^2}{\frac{1}{m}}) = \exp(-2 \epsilon^2 m) =: \frac{\delta}{2}\]
Wir haben $c_i = \frac{1}{m}$
\[\Rightarrow -2 \epsilon^2 m =  \ln (\frac{\delta}{2})\]
\[\Rightarrow - \epsilon^2 = \frac{\ln (\frac{\delta}{2})}{2m}\]
\[\epsilon = \frac{\sqrt{|\ln(\frac{\delta}{2})|}}{2m}\]
%Bild
$(**)$ Trick: $E[g(z)] = \frac{1}{m} \Sigma^m_{i=1} E[g(z'^{(i)})]$, wobei $s' = \{z'^{(1)}, \dots , z'^{m}\}$ Sample, das unabhängig gemäß $\varphi$ aus $z$ gezogen wurde.
\[E_s[\underset{g \in F}{\sup} (\frac{1}{m} \Sigma^m_{i=1} E[g(z'^{(m)})] - \frac{1}{m} \Sigma^m_{i=1} g(z^{(i)}))]\]