\newcommand{\xj}[1]{\phi_{x_j, y=#1}}

\section*{lecture 7 (25.04.2016}
\subsection{Naive Bayes (Bsp. Kontingenzanalyse)}
	\subsubsection*{Parameter}
		\[ P[y=1] = \phi_y \quad, y \in \{0,1\} Variates\]
		\[ P[x_i|y=1] = \phi_{x_i, y = 1} \quad, x \in \{0,1\}^n Covariates  \Rightarrow x_i \in \{0,1\}\]
		\[ P[x_i|y=0] = \phi_{x_i, y = 0}\]
		\[2n + 1 Parameter \]
	
	\subsubsection*{Maximum-Likelihood-Schätung}
	
		Daten: $ ((x^{(1)},y^{(1)}),\dots,(x^{(m)},y^{(m)}) ) $
		
		\[ \phi_0 = \frac{1}{m} \textcolor{blue}{\sum_{i = 1}^{m} y^{(i)}}\]
		
		\[ \xj{1} = \frac{\sum_{i=1}^{m} x_j ^{(i)} y^{(i)}}{\textcolor{blue}{\sum_{i=1}^{m}y^{(i)}}} \]
			
			\textcolor{blue}{Zählen wie oft y = 1 beobachtet wurde}
			
		\[ \xj{0} =  \frac{\sum_{i=1}^{m} x_j ^{(i)} (1- y^{(i)})}{\textcolor{cyan}{\sum_{i=1}^{m}(1- y^{(i)})}} \]
		
			\textcolor{cyan}{Zählen wie oft y = 0 beobachtet wurde}
			
		Vorhersage an Stelle $ x \in \{0,1\}^n $:
		
		\begin{framed}
			\[ y = argmax_{y' = 0,1} P[y'|x] \]
		\end{framed}
		
		%TODO bin mir nicht sicher ob $ P[y = 1|x] + P[y = 0|x] = 1$ richtig
		Dazu müssen wir ausrechnen: $ P[y = 1|x] $ und $ P[y = 0|x] $
		\textcolor{blue}{Wegen $ P[y = 1|x] + P[y = 0|x] = 1$ reicht es eine der Wahrscheinlichkeiten auszurechnen} 
		
		\[ P[y = 1|x] \textcolor{red}{=} \frac{P[y = 1|x]}{P[x]} \textcolor{red}{=} \frac{P[x| y = 1]P[y = 1]}{P[x]} \]
		
		\textcolor{red}{Bayes Theorem}
		
		\[ = \frac{P[y = 1] \prod_{i = 1}^{n} P[x_i | y = 1]}{P[y = 0] \prod_{i = 1}^{n} P[x_i | y = 0] + P[y = 1] \prod_{i = 1}^{n} P[x_i | y = 1] }\]
		\[ \frac{\phi_y \prod_{i = 1}^{n} \xj{1}^{x_i}(1- \xj{1})^{(1-x_i)}}{(1 - \phi_y) \prod_{i = 1}^{n} \xj{0}^{x_i}(1- \xj{0})^{(1-x_i)}} + \phi_y \prod_{i = 1}^{n} \xj{1}^{x_i}(1- \xj{1})^{(1-x_i)} \]
		
		\textcolor{red}{Beachte: Die Parameter $ \phi_y, \xj{0}, \xj{1} $ haben wir aus den Daten geschätzt}	
	
	\subsubsection*{Problem:}
		
		Falls feature $ x_i $ nicht in den Trainingsdaten auftaucht dann schätzen wir $ \xj{0} = \xj{1} = 0  $
	
	\subsubsection*{Lösungsidee:}
		
		Regularisierung, Laplace Smoothing
		
		\textcolor{blue}{Statt $ m $ Beobachtungen $ m + \underbrace{2n}_{\substack{\textcolor{red}{kuenstlich}}} $ Beobachtungen,wobei jedes feature einmal für $ y = 0 $ bzw. $ y = 1 $ künstlich beobachtet wurde.}
		
%			\begin{figure}
%				\centering
%				\includegraphics[width=0.7\linewidth]{graphs/plot7_1}
%				\caption{Hyperwürfel mit Hyperebene die Entscheidungsfunktion repräsentiert}
%			\end{figure}
		
		Naive Bayes Klassifikation ist wieder linear.
		
		\[ \phi_y \prod_{i = 1}^{n} \xj{1}^{x_i} (1 - \xj{1})^{(1 - x_j)}\]
		\[ = \exp(log(\phi_y) + \sum_{i = 1}^{n} x_i log(\xj{1}) + \sum_{i = 1}^{n} (1-x_i) log(1 - \xj{1})) \]

		\[ = \exp(\underbrace{\sum_{i = 1}^{n} \log(\frac{\xj{1}}{1- \xj{1}})}_{\substack{= X^T\Theta_i \quad, \Theta = (\Theta1,\dots,\Theta_n), \Theta_i = log(\frac{\xj{1}}{1-\xj{1}})}} + \underbrace{\log(\phi_y) + \sum_{i=1}^{n} \log(1 -\xj{1})}_{\substack{=\Theta_0}})\]
		
		\[ \Rightarrow P[y = 1| x] = \frac{1}{1 + \exp(- \tilde{\Theta} x)} \]
		\[ \Rightarrow P[y = 0| x] = 1- \frac{1}{1 + \exp(- \tilde{\Theta} x)} \]
		$ \Rightarrow $ Linearer Klassifikator (wie bei gauss'scher Diskriminanzanalyse) 
		
	\subsection*{Disrekte Auswahlprobleme (Viertes Einführungsbeispiel: Skalierung)}
		Daten: $ (((a^{(1)}, b^{(1)}), y^{m}),\dots, ((a^{(1)}, b^{(m)}),y^{(m)})) $
		
		\[ y^{(i)} \in \{0,1\} \quad, Variates\]
		\[ a^{(i)}, b^{(i)} \in A,\quad \text{wobei} \text{A eine endliche Menge von Optionen ist}\]
		
		\[ \text{Es gilt } y^{(i)} =
		\begin{cases}1 & a^{(i)} \succeq b^{(i)}\\0 & a^{(i)} \prec b^{(i)}
		\end{cases} \quad \textcolor{red}{\succ \text{Präferenzoperator}}\]
		

		
	\subsubsection*{Annahme}
		Es gibt Nutzenfunktionen $ u : A \rightarrow \mathbb{R} $
		\[ a \succeq b \text{genau dann, wenn } u(a) \geq u(b) \]
	\subsubsection*{Ziel}
		Schätze die Nutzenfunktion aus den Daten
	\subsubsection*{Ansatz}
		\[ u(a) = \underbrace{v(a)}_{\textcolor{red}{\text{deterministischer Anteil, \underline{soll geschätzt werden}}}} + \underbrace{\epsilon(a)}_{\textcolor{blue}{\text{probabilistisches Rauschen, \underline{wird modelliert}}}}\]
		
	\subsubsection*{Kurzschreibweise}
		\[ u(a) = u_a, v(a) = v_a, \epsilon(a)= \epsilon_a\]
		\[ P_{ab} = P[u_a \geq u_b] = P[v_a + \epsilon_a \geq v_b + \epsilon_b] =P[v_a - v_b \geq \underbrace{\textcolor{red}{\epsilon_b -\epsilon_a}]}_{\text{Zufallsvariable}}\]
				
	\subsubsection*{Modellierung von $ \epsilon_a $:}
	\subsubsection*{1. Thurstone Modell(Probit Methode)}
	
		$ \epsilon_a $ ist Gauss'sches / Normalverteiltes Rauschen
		\[ \Rightarrow \text{Dichte} p(\epsilon_a)= \frac{1}{2 \pi \sigma} \exp(\frac{1}{2}\frac{\epsilon_a^2}{2 \sigma^2}), \textcolor{red}{\text{ Varianz: } \sigma^2, \text{ Erwartungswerg: } 0}\]
		\[ \Rightarrow \text{Differenz } \epsilon_b - \epsilon_a \text{ ist wieder normalverteilt, wobei sich Erwartungswerte und Varanzen addieren} \]
	
		
	\subsubsection*{Es folgt:}
		\[ P_{ab}  = P[v_a -v_b \geq \epsilon_b - \epsilon_a] =  \frac{1}{2 \sqrt{\pi} \sigma} \int_{-\infty}^{v_a - v-b} \exp(-\frac{x^2}{4 \sigma^2}) dx\]
		\[ =: \Phi (\frac{v_a - v_b}{\sqrt{2} \sigma}) \quad \textcolor{red}{\text{Definition! } \Phi \text{-Funktion ist tabelliert}}\]
		\[ \Rightarrow v_a -v_b = \sqrt{2} \sigma \Phi^{-1} (p_{ab})\]
		
		\textcolor{red}{$ p_ab $ ist nicht bekannt, aber wir können $ p_{ab} $ schätzen durch $ t_{ab} $ }
		\[ \textcolor{red}{t_{ab} \frac{\text{\# Vergleiche von $ a $ und b, die $ a $ gewinnt}}{\text{\# Vergleiche von $ a $ und $ b $}}} \]
		
	\subsubsection*{Definiere:}
		\[ v_{ab} = \sqrt{2} \sigma \Phi^{-1} (t_{ab}) \text{\textcolor{blue}{Bemerkung: Wir können OBDA $ \sigma = 1 $ setzen}}\]
		
		D.h. wir haben einen Schätzwert $ v_{ab} $, aber eigentlich brauchen wir Schätzwerte für $ v_a $!
		\textcolor{blue}{Aufgabe: Ausrechnen der $ v_a $ aus den$  v_{ab} $ (später!)}
		
	\subsubsection*{2. Bradley Terry Modell (Logit Modell)}
		$ \epsilon_a $ ist Extremwert / Grumbel verteilt
		$ \Rightarrow \epsilon_b - \epsilon_a $ ist logistisch verteilt
	\subsubsection*{Es folgt:}
		\[ P_ab = P[v_a - v_b \geq \epsilon_b - \epsilon_a] = \frac{1}{1+ \exp(-(v_a + v_b))}\]
		\[ = \frac{\exp(v_a + v_b)}{\exp(v_a - v_b) +1} = \frac{\exp(v_a)}{\exp(v_a) + \exp(v_b)}  \]
		\[ \Rightarrow \frac{P_{ab}}{1- P_{ab}} = \frac{P_{ab}}{P_{ba} \frac{\frac{\exp(v_a)}{\exp(v_a) + \exp(v_b)}}{\frac{\exp(v_b)}{\exp(v_a) + \exp(v_b)}} \]
		\[ \frac{\exp(v_a)}{\exp(v_b)} = \exp(v_a - v_b) \]
		\[ \Rightarrow v_a - v_b = \log(\frac{P_{ab}}{1- P_{ab}}) \textcolor{red}{\text{$ P_{ab} $ ist nicht bekannt, dafür aber der Schätzwert $ t_{ab} $}}\]
		\[ v_{ab} = \log(\frac{t_{ab}}{1 - t_{ab}}) \]
		
		Wie im Probit Fall haben wir einen Schätzwert für die $ v_{ab} $ aus denen wir Schätzwerte für die $ v_a $ berechnen müssen
		
	\subsubsection*{}
	
	
	
